\documentclass{article}
\usepackage{graphicx} % Required for inserting images
\usepackage{mathtools}
\usepackage{parskip}
\usepackage{listings}
\usepackage{xcolor}
\usepackage{caption}

\definecolor{codegreen}{rgb}{0,0.6,0}
\definecolor{codegray}{rgb}{0.5,0.5,0.5}
\definecolor{codepurple}{rgb}{0.8,0,0.2}
\definecolor{backcolour}{rgb}{.95,.95,1}

\lstdefinestyle{mystyle}{
    backgroundcolor=\color{backcolour},   
    commentstyle=\color{codegreen},
    keywordstyle=\color{blue},
    numberstyle=\tiny\color{codegray},
    stringstyle=\color{codepurple},
    basicstyle=\ttfamily\footnotesize,
    breakatwhitespace=false,         
    breaklines=true,                 
    captionpos=b,                    
    keepspaces=true,                 
    numbers=left,                    
    numbersep=5pt,                  
    showspaces=false,                
    showstringspaces=false,
    showtabs=false,                  
    tabsize=2
}

\lstset{style=mystyle}

\title{Entropia de Dados}
\author{Isabela Santos \hspace{0.1cm} Julia Molitzas \hspace{0.1cm} Júlia Rodriguez}
\date{2023}


\begin{document}
\maketitle

\newpage

\section{Fórmula:}

Para calcular a entropia:

\[
\ H = - \sum_{x  \hspace{0.1cm} \in  \hspace{0.1cm} Classes}P(x)\hspace{0.1cm} . \hspace{0.1cm} log_2 \hspace{0.1cm} (P(x))\]\newline

\section{Aplicação dos Dados:}

O conjunto de dados inclui os 1000 filmes que fizeram mais sucesso no período de 2006 a 2016.\newline

As Classes são os gêneros dos filmes, que podem variar de "Sci-fi" a "Romance".\newline


Classes = \{Action, Adventure, Animation, Biography, Comedy, Crime, Drama, Fantasy, Horror, Music, Mystery, Romance, Sci-fi, Sport, Thriller, War, Western\}\newline 

Classe Action = 150 filmes \newline

Classe Adventure = 26 filmes \newline

Classe Animation = 49 filmes \newline

Classe Biography = 56 filmes \newline

Classe Comedy = 132 filmes \newline

Classe Crime = 53 filmes \newline

Classe Drama = 151 filmes \newline

Classe Fantasy = 55 filmes \newline

Classe Horror = 89 filmes \newline

Classe Music = 8 filmes \newline

Classe Mystery = 18 filmes \newline

Classe Romance = 70 filmes \newline

Classe  Sci-fi = 80 filmes \newline

Classe Sport = 4 filmes \newline

Classe Thriller = 52 filmes \newline

Classe War = 5 filmes \newline

Classe Western = 2 filmes \newline

\section{Probabilidades:\newline}

P(Classe Action) = $\frac{150}{1000}$ $\approx 0.150$ 

P(Classe Adventure) = $\frac{26}{1000}$ $\approx 0.026$ 

P(Classe Animation) = $\frac{49}{1000}$ $\approx 0.049$ 

P(Classe Biography) = $\frac{56}{1000}$ $\approx 0.056$ 

P(Classe Comedy) = $\frac{132}{1000}$ $\approx 0.132$ 

P(Classe Crime) = $\frac{53}{1000}$ $\approx 0.053$ 

P(Classe Drama) = $\frac{151}{1000}$ $\approx 0.151$ 

P(Classe Fantasy) = $\frac{55}{1000}$ $\approx 0.055$ 

P(Classe Horror) = $\frac{89}{1000}$ $\approx 0.089$  

P(Classe Music) = $\frac{8}{1000}$ $\approx 0.008$ 

P(Classe Mystery) = $\frac{18}{1000}$ $\approx 0.018$

P(Classe Romance) = $\frac{70}{1000}$ $\approx 0.070$ 

P(Classe  Sci-fi) = $\frac{80}{1000}$ $\approx 0.080$ 

P(Classe Sport) = $\frac{4}{1000}$ $\approx 0.004$

P(Classe Thriller) = $\frac{52}{1000}$ $\approx 0.052$

P(Classe War) = $\frac{5}{1000}$ $\approx 0.005$

P(Classe Western) = $\frac{2}{1000}$ $\approx 0.002$

\section{Aplicação da Fórmula:\newline}

\begin{align*}
    \ H = - \sum_{x  \hspace{0.1cm} \in  \hspace{0.1cm} \{1, 2, 3, 4, 5, 6, 7, 8, 9, 10, 11, 12, 13, 14, 15, 16, 17\}} P(x)\hspace{0.1cm} . \hspace{0.1cm} log_2 \hspace{0.1cm} (P(x))\newline
\end{align*}

\begin{align*}
    \begin{split}
        H &= - (0.150 \cdot \log_2 0.150) + (0.026 \cdot \log_2 0.026) + (0.049 \cdot \log_2 0.049) + (0.056 \cdot \log_2 0.056) \\
        &\quad + (0.132 \cdot \log_2 0.132) + (0.053 \cdot \log_2 0.053) + (0.151 \cdot \log_2 0.151) + (0.055 \cdot \log_2 0.055) \\
        &\quad + (0.089 \cdot \log_2 0.089) + (0.008 \cdot \log_2 0.008) + (0.018 \cdot \log_2 0.018) + (0.070 \cdot \log_2 0.070) \\
        &\quad + (0.080 \cdot \log_2 0.080) + (0.004 \cdot \log_2 0.004) + (0.052 \cdot \log_2 0.052) + (0.005 \cdot \log_2 0.005) \\
        &\quad + (0.002 \cdot \log_2 0.002)
    \end{split}
\end{align*}


\begin{align*}
\log_2 \, 0.150 &= \frac{\log 0.150}{0.3} \approx -2.74 & \log_2 \, 0.089 &= \frac{\log 0.089}{0.3} \approx -3.5 \\
\log_2 \, 0.026 &= \frac{\log 0.026}{0.3} \approx -5.2 & \log_2 \, 0.008 &= \frac{\log 0.008}{0.3} \approx -6.98 \\
\log_2 \, 0.049 &= \frac{\log 0.049}{0.3} \approx -4.36 & \log_2 \, 0.018 &= \frac{\log 0.018}{0.3} \approx -5.81 \\
\log_2 \, 0.056 &= \frac{\log 0.056}{0.3} \approx -4.17 & \log_2 \, 0.070 &= \frac{\log 0.070}{0.3} \approx -3.84 \\
\log_2 \, 0.132 &= \frac{\log 0.132}{0.3} \approx -2.93 & \log_2 \, 0.080 &= \frac{\log 0.080}{0.3} \approx -3.65 \\
\log_2 \, 0.053 &= \frac{\log 0.053}{0.3} \approx -4.25 & \log_2 \, 0.004 &= \frac{\log 0.004}{0.3} \approx -7.99 \\
\log_2 \, 0.151 &= \frac{\log 0.151}{0.3} \approx -2.73 & \log_2 \, 0.052 &= \frac{\log 0.052}{0.3} \approx -4.27 \\
\log_2 \, 0.055 &= \frac{\log 0.055}{0.3} \approx -4.49 & \log_2 \, 0.005 &= \frac{\log 0.005}{0.3} \approx -7.67 \\
\log_2 \, 0.002 &= \frac{\log 0.002}{0.3} \approx -8.99 \\
\end{align*}


\begin{align*}
    H &= - (0.150 \cdot (-2.74) + 0.026 \cdot (-5.2) + 0.049 \cdot (-4.36) + 0.056 \cdot (-4.17) \\
    &\quad + 0.132 \cdot (-2.93) + 0.053 \cdot (-4.25) + 0.151 \cdot (-2.73) + 0.055 \cdot (-4.49) \\
    &\quad + 0.002 \cdot (-8.99) + 0.089 \cdot (-3.5) + 0.008 \cdot (-6.98) + 0.018 \cdot (-5.81) \\
    &\quad + 0.070 \cdot (-3.84) + 0.080 \cdot (-3.65) + 0.004 \cdot (-7.99) + 0.052 \cdot (-4.27) \\
    &\quad + 0.005 \cdot (-7.67))
\end{align*}

$H = -(- 3.5911)$ \newline
$H \approx 3.59$\newline

\section{Cálculo da entropia máxima:}

\begin{align*}
    Hmax &= \log_2 17 \\
    Hmax &= \frac{1.23}{0.3} \\
    Hmax &= 4.1 \\
\end{align*}

\section{Conclusão:}
Os dados apresentados tem o valor da entropia próximo ao valor da entropia máxima, assim possuindo um alto grau de aleatoriedade, ou seja, os dados se distanciam da distribuição uniforme.\newline

\section{Programação Python:}
\subsection{Código:}

\begin{lstlisting}[language=Python]
#importacao do bando de dados
import pandas as pd

#ler o arquivo do Excel
dados_excel = pd.read_excel(r"C:\Users\Fatec\Downloads\filmes-dataset.xlsx")


print("Filmes:")

#acessando a coluna especifica dos generos pelo numero do indice
g = 0
while g < 17:
    nome_coluna = dados_excel.iloc[:, g]
    g += 1
    for valor in nome_coluna:
        if not pd.isna(valor): 
            print(valor) 
    print()
    
    
print("Tamanho das Colunas de Generos:")
    
for coluna in dados_excel.columns:
    if dados_excel[coluna].count() > 0:
        print("Coluna:", coluna)
        print("Tamanho:", dados_excel[coluna].count())
        print()

#calculo das probabilidades
total_elementos = 0
for coluna in dados_excel.columns:
    if dados_excel[coluna].count() > 0: 
        total_elementos += dados_excel[coluna].count() 
print("Valor total: ", total_elementos) 
print()

probabilidades = []
for coluna in dados_excel.columns:
    if dados_excel[coluna].count() > 0:
        p = dados_excel[coluna].count() / total_elementos
        probabilidades.append(p)
        print(f"Probabilidades: {p:.3f}") 
print()


#calculo da entropia

import math

def calcular_entropia(probabilidades):
    entropia = 0
    for p in probabilidades:
        entropia += p * math.log2(p)
    entropia = -entropia
    return entropia

entropia_resultante = calcular_entropia(probabilidades)
print(f"Entropia: {entropia_resultante:.2f}")

#entropia maxima

entropia_max = math.log2(17)
print(f"Entropia maxima: {entropia_max:.2f} ")

\end{lstlisting}

\newpage

\subsection{Resultado no Console:\newline }

Exemplo de como são impressos os filmes (filmes de Thriller, War, Western):
\begin{figure}[ht]
  \centering
  \includegraphics[width=1.2\textwidth]{Exemplo Filmes Console.png}
  \label{fig:imagem}
\end{figure}

Continuação do Console após impressão dos filmes:

\begin{verbatim}

Tamanho das Colunas de Gêneros:
Coluna: ACTION
Tamanho: 150

Coluna: ADVENTURE
Tamanho: 26

Coluna: ANIMATION
Tamanho: 49

Coluna: BIOGRAPHY 
Tamanho: 56

Coluna: COMEDY
Tamanho: 132

Coluna: CRIME
Tamanho: 53

Coluna: DRAMA 
Tamanho: 151

Coluna: FANTASY 
Tamanho: 55

Coluna: HORROR 
Tamanho: 89

Coluna: MUSIC
Tamanho: 8

Coluna: MYSTERY 
Tamanho: 18

Coluna: ROMANCE 
Tamanho: 70

Coluna: SCI-FI
Tamanho: 80

Coluna: SPORT 
Tamanho: 4

Coluna: THRILLER 
Tamanho: 52

Coluna: WAR
Tamanho: 5

Coluna: WESTERN
Tamanho: 2

Valor total:  1000

Probabilidades: 0.150
Probabilidades: 0.026
Probabilidades: 0.049
Probabilidades: 0.056
Probabilidades: 0.132
Probabilidades: 0.053
Probabilidades: 0.151
Probabilidades: 0.055
Probabilidades: 0.089
Probabilidades: 0.008
Probabilidades: 0.018
Probabilidades: 0.070
Probabilidades: 0.080
Probabilidades: 0.004
Probabilidades: 0.052
Probabilidades: 0.005
Probabilidades: 0.002

Entropia: 3.59
Entropia máxima: 4.09 

\end{verbatim}


\end{document}
